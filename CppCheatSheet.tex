\documentclass[10pt,landscape]{article}
\usepackage{multicol}
\usepackage{calc}
\usepackage{ifthen}
\usepackage[landscape]{geometry}
\usepackage{hyperref}

% To make this come out properly in landscape mode, do one of the following
% 1.
%  pdflatex latexsheet.tex
%
% 2.
%  latex latexsheet.tex
%  dvips -P pdf  -t landscape latexsheet.dvi
%  ps2pdf latexsheet.ps


% If you're reading this, be prepared for confusion.  Making this was
% a learning experience for me, and it shows.  Much of the placement
% was hacked in; if you make it better, let me know...


% 2008-04
% Changed page margin code to use the geometry package. Also added code for
% conditional page margins, depending on paper size. Thanks to Uwe Ziegenhagen
% for the suggestions.

% 2006-08
% Made changes based on suggestions from Gene Cooperman. <gene at ccs.neu.edu>


% To Do:
% \listoffigures \listoftables
% \setcounter{secnumdepth}{0}


% This sets page margins to .5 inch if using letter paper, and to 1cm
% if using A4 paper. (This probably isn't strictly necessary.)
% If using another size paper, use default 1cm margins.
\ifthenelse{\lengthtest { \paperwidth = 11in}}
	{ \geometry{top=.5in,left=.5in,right=.5in,bottom=.5in} }
	{\ifthenelse{ \lengthtest{ \paperwidth = 297mm}}
		{\geometry{top=1cm,left=1cm,right=1cm,bottom=1cm} }
		{\geometry{top=1cm,left=1cm,right=1cm,bottom=1cm} }
	}

% Turn off header and footer
\pagestyle{empty}
 

% Redefine section commands to use less space
\makeatletter
\renewcommand{\section}{\@startsection{section}{1}{0mm}%
                                {-1ex plus -.5ex minus -.2ex}%
                                {0.5ex plus .2ex}%x
                                {\normalfont\large\bfseries}}
\renewcommand{\subsection}{\@startsection{subsection}{2}{0mm}%
                                {-1explus -.5ex minus -.2ex}%
                                {0.5ex plus .2ex}%
                                {\normalfont\normalsize\bfseries}}
\renewcommand{\subsubsection}{\@startsection{subsubsection}{3}{0mm}%
                                {-1ex plus -.5ex minus -.2ex}%
                                {1ex plus .2ex}%
                                {\normalfont\small\bfseries}}
\makeatother

% Define BibTeX command
\def\BibTeX{{\rm B\kern-.05em{\sc i\kern-.025em b}\kern-.08em
    T\kern-.1667em\lower.7ex\hbox{E}\kern-.125emX}}

% Don't print section numbers
\setcounter{secnumdepth}{0}


\setlength{\parindent}{0pt}
\setlength{\parskip}{0pt plus 0.5ex}


% -----------------------------------------------------------------------

\begin{document}

\raggedright
\footnotesize
\begin{multicols}{3}


% multicol parameters
% These lengths are set only within the two main columns
%\setlength{\columnseprule}{0.25pt}
\setlength{\premulticols}{1pt}
\setlength{\postmulticols}{1pt}
\setlength{\multicolsep}{1pt}
\setlength{\columnsep}{2pt}

\begin{center}
     \Large{\textbf{LARICS Code Standard Cheat Sheet}} \\
\end{center}

\section{Naming conventions}
\begin{tabular}{@{}ll@{}}
	Types           & \verb!Line!, \verb!SavingsAccount! \\
	Variable names  & \verb!line!, \verb!savings_account! \\
	Named constants & \verb!MAX_ITERATIONS!, \verb!PI! \\
	Functions       & \verb!getName()!, \verb!computeTotalWidth()! \\
	Namespaces      & \verb!model::analyzer!, \verb!std::cout()! \\
	Templates       & \verb!template<class T>! // use one letter\\
	Abreviations	& \verb!exportHtmlSource();! // NOT: \verb!HTML! \\
	and acronyms    & \verb!openDvd();! // NOT: \verb!openDVD();! \\
	% Section 3.10??
	Private class   & \verb!private:! // Should go with underscore \\
	members         & \verb!  int length_;! \\
	                & \verb!protected:! // Underscore \\
	                & \verb!  int width_;! \\
	Generic vars    & \verb!Point start_point, end_point;! \\
	Method name     & \verb!line.getLength();! \\ 
	                & \hspace{0.5cm} NOT: \verb!line.getLineLength();! \\	
\end{tabular}

\subsection{Specific naming conventions}
\begin{tabular}{@{}ll@{}}
	Get and set     & \verb!employee.getName();! \\
	                & \verb!employee.setName(someName);! \\
	                & \verb!matrix.getElement(2, 4);! \\
	                & \verb!matrix.setElement(2, 4, value);! \\
	Term compute    & \verb!value_set->computeAverage();! \\
	                & \verb!matrix->computeInverse();! \\
	Term find       & \verb!vertex.findNearest();! \\
	                & \verb!matrix.findMinElement();! \\
	Term initialize & \verb!printer.initializeFontSet();! \\
	                & \\
	GUI with compo- & \verb!main_window!, \verb!width_scale!, \verb!login_text!, \\
	nent suffix     & \verb!left_scrollbar!, \verb!min_label!, \verb!yes_toggle! \\
	                & \\
	Plurals         & \verb!vector<Point> points;!, \verb!int values[];! \\
	Objects number  & \verb!n_points!, \verb!n_lines! // Use n prefix\\
	Entity number   & \verb!table_no!, \verb!employee_no! // Use No suffix\\
	Iterators should& \verb!for (int i=0; i < n_tables); i++) {!\\ 
	be i, j, k...   & \hspace{0.5cm} \verb!*statements*! \\
	                & \verb!}!\\
	is prefix for   & \verb!is_set!, \verb!is_visible!, \verb!is_finished!, \\
	bool methods    & \verb!is_found!, \verb!is_open! \\
	                & \\
	% Section 3.27?
	Avoid           & \verb!computeAverage();! // NOT: \verb!compAvg();! \\
	abbreviations   & \\ 
	Exceptions for  & \verb!HypertextMarkupLanguage! instead of \verb!html! \\
	known phrases   & \verb!CentralProcessingUnit! instead of \verb!cpu! \\
	                & \verb!PriceEarningRatio! instead of \verb!pe! \\
	                & etc. \\
	Avoid naming    & \verb!Line *line;! // NOT: \verb!Line *pline;! \\
	pointers        &                    // NOT: \verb!Line *line_ptr;! \\
	                & \\
	Avoid negated   & \verb!bool is_error;! // NOT: \verb!is_no_error;! \\
	bool            & \verb!bool is_found;! // NOT: \verb!is_not_found;! \\
	                & \\
\end{tabular} \\

\begin{tabular}{@{}ll@{}}
	Prefix common   & \verb!enum Color{! \\
	type in enum    & \hspace{0.2cm} \verb!COLOR_RED! \\
	                & \hspace{0.2cm} \verb!COLOR_GREEN! \\
	                & \hspace{0.2cm} \verb!COLOR_BLUE! \\
	                & \verb!};! \\
	                & \\
	Exception class & \verb!class AccessException! \\
	suffix          & \verb!{! \\
	                & \hspace{0.2cm} \verb!:! \\
	                & \verb!}! \\
\end{tabular}

\section{Files}
\subsection{Source Files}
\begin{tabular}{@{}ll@{}}
	Header .h and  & \verb!MyClass.cpp!, \verb!MyClass.h! \\
	source .cpp    & API in .h, definitions in .cpp file \\
	               & \\
	Aligned        & \verb!total_sum = a + b + c + ! \\
	split          & \verb!           d + e;! \\
	               & \verb!function(param1, param2,! \\
	               & \verb!         param3);! \\
	               & \verb!for (int table_no = 0; table_no < n_tables;! \\
	               & \verb!     table_no += table_step)! \\
\end{tabular}

\subsection{Include files and statements}
\begin{tabular}{@{}ll@{}}
	Include guards & \verb!#ifndef COM_COMPANY_MODULE_CLASSNAME_H! \\
	               & \verb!#define COM_COMPANY_MODULE_CLASSNAME_H! \\
	               & \verb!...! \\
	               & \verb!#endif COM_COMPANY_MODULE_CLASSNAME_H! \\
	               & \\
	Group and sort & \verb!#include<fstream>! \\
	include        & \verb!#include<iomanip>! \\
	statements     & \\
	               & \verb!#include<qt/qbutton.h>! \\
	               & \verb!#include<qt/qtextfield.h>! \\
	               & \\
	               & \verb!#include "com/company/ui/BoxDialog.h"! \\
	               & \verb!#include "com/company/ui/MainWindow.h"! \\
	               & \\
\end{tabular}

\section{Statements}
\subsection{Types}
\begin{tabular}{@{}ll@{}}
	Class layout  & \verb!Class MyClass! \\
	              & \verb!{! \\
	              & \hspace{0.2cm} \verb!public:! \\
	              & \hspace{0.2cm} \verb!protected:! \\
	              & \hspace{0.2cm} \verb!private:! \\
	              & \verb!};! \\
	              & \\
	Use explicit  & \verb!float_value = static_cast<float>(int_value);! \\
	conversions   & NOT: \verb!float_value = int_value;!\\
\end{tabular}

\subsection{Variables}
- initialize variables upon declaration, if unable leave it uninitialized rather than initialized to some phony value. \\
- no global variables or functions \\
\begin{tabular}{@{}ll@{}}
	Pointers and  & \verb!float *x;! // NOT: \verb!float* x;! \\
	references    & \verb!int &x;! // NOT: \verb!int& x;! \\
	              & \\
	Implicit test & \texttt{if (n\_lines != 0)} // NOT: \verb!if (n_lines)! \\
	for 0         & \texttt{if (value != 0.0)} // NOT: \verb!if (value)! \\
\end{tabular}

\subsection{Loops}
\begin{tabular}{@{}ll@{}}
	Only put loop & \verb!sum = 0;! \\
	control       & \verb!for (i = 0; i < 100; i++)! \\
	statements in & \hspace{0.2cm} \verb!sum += value[i]! \\
	for loop      & \\
	NOT:          & \verb!for (i = 0, sum = 0; i < 100; i++)! \\
	              & \hspace{0.2cm} \verb!sum += value[i]! \\
	              & \\
	Init loop vars& \verb!is_done = false;! \\
	just before   & \texttt{while (!is\_done) \{} \\ 
	loop          & \hspace{0.2cm} \verb!*statements*! \\
	              & \verb!}! \\
	              & \\
	Infinite loop & \verb!while(true)! // Use only this form \\
\end{tabular} \\

-avoid \verb!do-while! loops when possible \\
-avoid \verb!break! and \verb!continue! when possible \\

\subsection{Conditionals}
-avoid complex conditional expressions. Use temporary \verb!bool! variables instead:
\begin{verbatim}
	bool is_finished = (element_no < 0) || 
	    (element_no > max_element);
	bool is_repeated_entry = element_no == last_element;
	if (is_finished || is_repeated_entry) {
  	  :
	}
\end{verbatim}

\begin{tabular}{@{}ll@{}}
	Nominal case  & \verb!bool is_ok = readFile(file_name);! \\
	and exception & \verb!if (is_ok) {! \\
	              & \hspace{0.2cm} \verb!*nominal case*! \\
	              & \verb!}! \\
	              & \verb!else {! \\
	              & \hspace{0.2cm} \verb!*exception*! \\
	              & \verb!}! \\
	              & \\
	Allow line    & \verb!if (is_done)! // NOT: \verb!if (is_done) doCleanup();! \\
	break         & \hspace{0.2cm} \verb!doCleanup();! \\
	              & \\
	Avoid         & \verb!File *file_handle = open(file_name, "w");! \\
	executable    & \texttt{if (!file\_handle)\{\}} \\
	statements    & // NOT: \\
	in conditions & \texttt{if (!(file\_handle = open(file\_name, "w"))) \{\}} \\
	              & \\
\end{tabular}

\subsection{Miscellaneous}
-numbers other than 0 and 1 should be named constants.\\

\begin{tabular}{@{}ll@{}}
	Floating point& \verb!double total = 0.0;! // NOT: \verb!double total = 0;! \\
	notation      & \verb!double sum = 3.0e8;! // NOT: \verb!double sum = 3e8;! \\
	              & \verb!double sum = (a + b) * 10.0;! \\
	              & \verb!double total = 0.5;! // NOT: \verb!double total = .5;! \\
	              & \\
\end{tabular}
-never use \verb!goto! statement \\
-use \verb!0! instead of \verb!NULL! \\

\section{Layout and Comments}
\subsection{Layout}
-basic indentation should be 2. \\

\begin{tabular}{@{}ll@{}}
	Loops: while  & \texttt{while (!done) \{ } \\
	              & \hspace{0.2cm} \verb!doSomething();! \\
	              & \hspace{0.2cm} \verb!done = moreToDo();! \\
	              & \verb!}! \\
	              & \\
	      do-while& \verb!do {! \\
	              & \verb!*statements*;! \\
	              & \verb!} while (condition);! \\
	              & \\
	       for    & \verb!for (initialization; condition; update) {! \\
	              & \hspace{0.2cm} \verb!*statements*;! \\
	              & \verb!}! \\
	              & \\
	If statements & \verb!if (condition) {! \\
	              & \hspace{0.2cm} \verb!*statements*;! \\
	              & \verb!}! \\
	              & \verb!else if (condition) {! \\
	              & \hspace{0.2cm} \verb!*statements*;! \\
	              & \verb!}! \\
	              & \verb!else {! \\
	              & \hspace{0.2cm} \verb!*statements*;! \\
	              & \verb!}! \\
	              & \\
	One statement & \verb!if (condition)! \\
	in if         & \hspace{0.2cm} \verb!*statement*;! \\
	              & \\
	Switch case   & \verb!switch (condition) {! \\
	              & \hspace{0.2cm} \verb!case A :! \\
	              & \hspace{0.4cm} \verb!*statements*;! \\
	              & \hspace{0.4cm} \verb!// Fallthrough! \\
	              & \\
	              & \hspace{0.2cm} \verb!case B :! \\
	              & \hspace{0.4cm} \verb!*statements*;! \\
	              & \hspace{0.4cm} \verb!break;! \\
	              & \\
	              & \hspace{0.2cm} \verb!default : ! \\
	              & \hspace{0.4cm} \verb!*statements*;! \\
	              & \hspace{0.4cm} \verb!break;! \\
	              & \verb!}! \\
	              & \\
	Functions and & \verb!void someMethod()! \\
	methods       & \verb!{! \\
	              & \hspace{0.2cm} \verb!*function code*;! \\
	              & \verb!}! \\
	              & \\
	 Try-catch    & \verb!try {! \\
	              & \hspace{0.2cm} \verb!*statements*;! \\
	              & \verb!}! \\
	              & \verb!catch (Exception& exception) {! \\
	              & \hspace{0.2cm} \verb!*statements*;! \\
	              & \verb!}! \\
	              & \\
\end{tabular}

\begin{tabular}{@{}ll@{}}
	Class layout  & \verb!Class MyClass! \\
	              & \verb!{! \\
	              & \hspace{0.2cm} \verb!public:! \\
	              & \hspace{0.2cm} \verb!protected:! \\
	              & \hspace{0.2cm} \verb!private:! \\
	              & \verb!};! \\
	              & \\
	Return type   & \verb!void! \\
	can be put    & \verb!MyClass::myMethod(void)! \\
	above function& \verb!{! \\
	              & \hspace{0.2cm} \verb!*class body*! \\
	              & \verb!}! \\
	              & \\
\end{tabular}

\subsection{White Space}
- Conventional operators should be surrounded by a space character. \\
- C++ reserved words should be followed by a white space. \\
- Commas should be followed by a white space. \\
- Colons should be surrounded by white space. \\
- Semicolons in for statments should be followed by a space character. \\


\begin{tabular}{@{}ll@{}}
	Operators     & \verb!a = (b + c) * d;! // NOT: \verb!a=(b+c)*d;! \\
	Loops         & \verb!while (true)! // NOT: \verb!while(true)! \\
	              & \verb!for (i = 0; i < 4; i++)! \\ 
	              & NOT: \verb!for(i=0;i<4;i++)!\\
	Functions     & \verb!min(a, b, c, d);! // NOT \verb!min(a,b,c,d);! \\
	Case          & \verb!case 100 : ! // NOT \verb!case 100:! \\
	
\end{tabular}

\subsubsection{Separating logical units}
- Use one blank space to separate, multiple if it enhances readability
\begin{verbatim}
Matrix4x4 matrix = new Matrix4x4();

double cosAngle = Math.cos(angle);
double sin_angle = Math.sin(angle);

matrix.setElement(1, 1,  cos_angle);
matrix.setElement(1, 2,  sin_angle);
matrix.setElement(2, 1, -sin_angle);
matrix.setElement(2, 2,  cos_angle);

multiply(matrix);

\end{verbatim}

\begin{verbatim}
if      (a == low_value)    compueSomething();
else if (a == medium_value) computeSomethingElse();
else if (a == high_value)   computeSomethingElseYet();

value = (potential        * oil_density)   / constant1 +
        (depth            * water_density) / constant2 +
        (zCoordinateValue * gas_density)   / constant3;

min_position     = computeDistance(min,     x, y, z);
average_position = computeDistance(average, x, y, z);

switch (value) {
  case PHASE_OIL   : strcpy(phase, "Oil");   break;
  case PHASE_WATER : strcpy(phase, "Water"); break;
  case PHASE_GAS   : strcpy(phase, "Gas");   break;
}
\end{verbatim}


\end{multicols}
\end{document}
